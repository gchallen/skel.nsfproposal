% 16 Nov 2010 : GWA : Well broken-down set of research tasks, objectives,
%               expected results, timetable.

\section{Research Tasks}
\label{section-researchtasks}

\subsection{Overview and Approach}
\label{subsection-overviewandapproach}

\subsection{Methods, Results, Contingency, Assessment}
\label{subsection-methodsresults}

%\begin{researchtasks}
%\end{researchtasks}

%%%%%%%%%%%%%%%%%%%%%%%%%%%%%%%%%%%%%%%%%%%%%%%%%%%%%%%%%%%%%%%%%%%%%%%%%%
% 28 Nov 2010 : GWA : Task.

%\item \textbf{Simulator augmentation} (from Q\ref{Q1}, Q\ref{Q2}, Q\ref{Q3}):
%Extend the M5 simulator~\cite{m5-ieeemicro} with power models.
%\label{R1}
%
%\uline{\textbf{\textsc{Methods} (Year 1):}}
%\begin{researchmethods}
%\item Measure power states for target components.
%\item Map the state of components simulated in M5 on to power model to produce
%component-by-component breakdown.
%\item Implement simple heterogeneous system in M5: two processors, two memory
%banks, two stable storage devices and two radios, each with different
%power-performance tradeoffs.
%\item Develop device interface to expose input power availability to the
%Linux kernel.
%\end{researchmethods}
%
%\uline{\textbf{\textsc{Expected Results:}}} The output of this task will be a
%augmented version of M5 suitable for simulating the systems we propose to
%study. It will accurately model the power consumption of an ensemble of
%hardware components, and include support for enabling and disabling
%components as input power fluctuates.
%
%\uline{\textbf{\textsc{Risk} \textit{(Contingency):}}} Developing accurate
%power models may be difficult for some components \textit{(simpler power
%models or components that can be more easily modeled can be used)}; the M5
%simulator may not produce accurate results \textit{(another simulator will be
%used instead)}.
%
%\uline{\textbf{\textsc{Assessment:}}} Accuracy of the power models can be
%verified empirically.

